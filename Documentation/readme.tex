\documentclass[11pt]{article}
\usepackage{amsmath}
\usepackage{amssymb}
\usepackage{makeidx}
\usepackage{graphicx}
\usepackage[margin=0.75in]{geometry}
\usepackage{setspace}
\usepackage{pdfpages}
\usepackage{hyperref}
\hypersetup{
    colorlinks=true, %set true if you want colored links
    linktoc=all,     %set to all if you want both sections and subsections linked
    linkcolor=black,  %choose some color if you want links to stand out
}
\newcommand{\be}{\begin{enumerate}}
\newcommand{\ee}{\end{enumerate}}
\title{README: Software for `Convex Denoising using Non-Convex Tight Frame Regularization'}
\author{Ankit Parekh\footnote{Department of Mathematics, School of Engineering, New York University (ankit.parekh@nyu.edu) \url{https://sites.google.com/a/nyu.edu/ankit-parekh}}}
\begin{document}
\maketitle


\begin{center}
\tableofcontents
\end{center}

\clearpage
\section{Introduction}
This is a readme file for the MATLAB code included with the paper `Convex Denoising using Non-Convex Tight Frame Regularization'. Jump to Section \ref{Examples} for examples. 

\section{Description of Core Functions}
\subsection{bp\textunderscore ncvx.m}
Signal denoising using non-convex tight frame regularization. 
\subsubsection{Usage}
[x,cost,err] = bp\textunderscore ncvx(y,A,AH,lam,a,mu,Nit,pen)
\\The matrices A,AH are function handles. 
\subsubsection{Inputs}
\begin{verbatim}
    y - Input signal
    A - Tight-frame, i.e., A^H * A = rI, r > 0
    AH - Conjugate transpose of A
    lam - Regularization parameter
    a - degree of non-convexity. Note: a < 1/(r*lam)
    mu - Augmented Lagrangian parameter for ADMM (mu > 1/r)
    Nit - Number of Iterations
    pen - Regularizer to be used 
            a. Logarithmic ('log')
            b. Rational    ('rat')
            c. Arctangent  ('atan')
            d. L1 norm     ('l1')
\end{verbatim}
\subsubsection{Outputs}
\begin{verbatim}
 	  x - Denoised signal
    cost - Cost function history
    err - Error when using variable splitting, i.e., ||u-Ax||_2^2   
\end{verbatim}

\subsection{bp\textunderscore ncvx2DCWT.m}
Convex denoising of 2D image using non-convex tight frame regularization with the 2D dual tree complex wavelet transform. 
\subsubsection{Usage}
x = bp\textunderscore ncvx2DCWT(y,A,AH,J,lam,a,mu,Nit,pen)
\subsubsection{Inputs}
\begin{verbatim}
    y - Input image
    A - Forward transform (Undecimated Wavelet transform)
    AH - Inverse transform
    J - Number of scales
    lam - Regularization parameter (vector)
    a - Degree of non-convexity (a < 1/(r*lam))
    mu - Augmented Lagrangian parameter (mu > 1/r)
    Nit - Number of iterations
    pen - Regularizer to be used 
            a. Logarithmic ('log')
            b. Rational    ('rat')
            c. Arctangent  ('atan')
            d. L1          ('l1')
\end{verbatim}

\subsubsection{Outputs}
\begin{verbatim}
    	x - denoised image
\end{verbatim}

\subsection{bp\textunderscore ncvxUDWT.m}
1D signal denoising using non-convex regularization with the undecimated wavelet transform
\subsubsection{Usage}
[x,cost] = bp\textunderscore ncvxUDWT(y,A,AH,J,lam,a,mu,Nit,pen)
\subsubsection{Inputs}
\begin{verbatim}
    y - Input Signal
    A - Forward transform (Undecimated Wavelet transform)
    AH - Inverse transform
    J - Number of scales
    lam - Regularization parameter (vector)
    a - Degree of non-convexity (a_i < 1/lam_i)
    mu - Augmented Lagrangian parameter (mu > 1/r)
    Nit - Number of iterations
    pen - Regularizer to be used 
            a. Logarithmic ('log')
            b. Rational    ('rat')
            c. Arctangent  ('atan')
            d. L1          ('l1')
\end{verbatim}
\subsubsection{Outputs}
\begin{verbatim}
    x - Denoised Signal
    cost - Cost function history
\end{verbatim}
\section{Matlab Demo}
\label{Examples}
\subsection{1D signal denoising}
Need to have ADOBE Reader installed. Else open the file `demo1D.pdf' manually. \\
\href{run:demo1D.pdf}{1D signal denoising MATLAB demo}
\subsection{2D image denoising}

Need to have ADOBE Reader installed. Else open the file `demo2D.pdf' manually. \\
\href{run:demo2D.pdf}{2D Image denoising MATLAB demo}



\end{document}